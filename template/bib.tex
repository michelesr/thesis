\begin{thebibliography}{999}

\bibitem{protractor}
\raggedright
{\em Protractor - end to end testing for AngularJS},
Available: https://angular.github.io/protractor/

\bibitem{docker-install}
Docker Inc. © 2015, "Supported Installation",
{\em Docker Documentation},
Available: https://docs.docker.com/installation/

\bibitem{docker-hub}
Docker Inc. © 2015,
{\em Docker Hub Registry - Repositories of Docker Images},
Available: https://hub.docker.com/

\bibitem{docker-compose}
\raggedright
Docker Inc. © 2015,
{\em Docker Compose},
Available: https://www.docker.com/docker-compose

\bibitem{docker-content-trust}
\raggedright
Docker Inc. © 2015,
{\em Introducing Docker Content Trust},
Available: https://blog.docker.com/2015/08/content-trust-docker-1-8/

\bibitem{docker-toolbox}
\raggedright
Docker Inc. © 2015,
{\em Docker Toolbox},
Available: https://www.docker.com/toolbox

\bibitem{xp-ci}
\raggedright
Don Wells © 1997-1999, All rights reserved, "Integrate Often",
{\em Extreme Programming},
Available: http://www.extremeprogramming.org/rules/integrateoften.html

\bibitem{unix}
\raggedright
Eric Steven Raymond © 2013, CC-BY-ND 1.0, "Chapter 1 - Philosophy",
{\em Basics of the Unix Philosophy},
Available: http://www.faqs.org/docs/artu/ch01s06.html

\bibitem{node-to-iojs}
\raggedright
Galina Pankratova, InfoWorld Inc. © 1994-2015, All rights reserved, 4 December 2014,
{\em Why io.js decided to fork Node.js},
Available: http://www.infoworld.com/article/2855057/application-development/why-iojs-decided-to-fork-nodejs.html

\bibitem{two-way-binding}
\raggedright
Google Inc. © 2010-2015, CC-BY 3.0, "Step 4: Two-Way Data Binding",
{\em AngularJS Tutorial},
Available: https://docs.angularjs.org/tutorial/step\_04

\bibitem{ng-model}
\raggedright
Google Inc. © 2010-2015, CC-BY 3.0, "ngModel directive",
{\em AngularJS API Reference},
Available: https://docs.angularjs.org/api/ng/directive/ngModel

\bibitem{ng-bind}
\raggedright
Google Inc. © 2010-2015, CC-BY 3.0, "ngBind directive",
{\em AngularJS API Reference},
Available: https://docs.angularjs.org/api/ng/directive/ngBind

\bibitem{ng-repeat}
\raggedright
Google Inc. © 2010-2015, CC-BY 3.0, "ngRepeat directive",
{\em AngularJS API Reference},
Available: https://docs.angularjs.org/api/ng/directive/ngRepeat

\bibitem{agile-manifesto}
\raggedright
Kent Beck, Mike Beedle, Arie van Bennekum, Alistair Cockburn, Ward Cunningham,
Martin Fowler, James Grenning, Jim Highsmith, Andrew Hunt, Ron Jeffries, Jon
Kern, Brian Marick, Robert C. Martin, Steve Mellor, Ken Schwaber, Jeff
Sutherland, Dave Thomas, © 2001,
{\em Manifesto for Agile Software Development},
Available: http://www.agilemanifesto.org/

\bibitem{agile-principles}
\raggedright
Kent Beck, Mike Beedle, Arie van Bennekum, Alistair Cockburn, Ward Cunningham,
Martin Fowler, James Grenning, Jim Highsmith, Andrew Hunt, Ron Jeffries, Jon
Kern, Brian Marick, Robert C. Martin, Steve Mellor, Ken Schwaber, Jeff
Sutherland, Dave Thomas, © 2001,
{\em Principles behind the Agile Manifesto},
Available: http://www.agilemanifesto.org/principles.html

\bibitem{cloudflare-ecdsa}
\raggedright
Nick Sullivan, Cloudflare Inc. © 2015, 10 March 2014, "ECDSA vs RSA",
{\em ECDSA: The digital signature algorithm of a better internet},
Available: https://blog.cloudflare.com/ecdsa-the-digital-signature-algorithm-of-a-better-internet/

\bibitem{martinfowler-ci}
\raggedright
Martin Fowler ©, 01 May 2006,
{\em Continuous Integration},
Available: http://www.martinfowler.com/articles/continuousIntegration.html

\bibitem{martinfowler-ms}
\raggedright
Martin Fowler ©, 10 March 2014,
{\em Microservices},
Available: http://martinfowler.com/articles/microservices.html

\bibitem{mdn-map}
\raggedright
Mozilla Developer Network and individual Contributors © 2005-2015, CC-BY-SA 2.5, "Array.prototype.map()",
{\em Web Technology for developers},
Available: https://developer.mozilla.org/en-US/docs/Web/JavaScript/Reference/Global\_Objects/Array/map

\bibitem{wikipedia-docker}
\raggedright
Wikimedia Foundation Inc. and Contributors ©, CC-BY-SA 3.0,
last modified on 18 August 2012, "Docker (software)",
{\em Wikipedia, The Free Encyclopedia},
Available: https://en.wikipedia.org/wiki/Docker\_(software)

\end{thebibliography}
